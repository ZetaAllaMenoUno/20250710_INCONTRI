% Options for packages loaded elsewhere
\PassOptionsToPackage{unicode}{hyperref}
\PassOptionsToPackage{hyphens}{url}
%
\documentclass[
  12pt,
]{article}
\usepackage{amsmath,amssymb}

% Supporto per caratteri Unicode speciali (es. Zero-width non-joiner)
% Manteniamo solo la parte per motori moderni come XeLaTeX.

% --- Miglioramenti Tipografici (Opzionali ma raccomandati) ---

% Usa virgolette dritte "..." invece di quelle curve nei blocchi di codice verbatim.
\IfFileExists{upquote.sty}{\usepackage{upquote}}{}

% Pacchetto per la micro-tipografia: migliora drasticamente la giustificazione
% del testo e l'aspetto generale del paragrafo.
\IfFileExists{microtype.sty}{%
  \usepackage[]{microtype}
  \UseMicrotypeSet[protrusion]{basicmath}
}{}




\usepackage{xcolor}

\usepackage{color}
\usepackage{fancyvrb}
\newcommand{\VerbBar}{|}
\newcommand{\VERB}{\Verb[commandchars=\\\{\}]}
\DefineVerbatimEnvironment{Highlighting}{Verbatim}{commandchars=\\\{\}}
% Add ',fontsize=\small' for more characters per line
\newenvironment{Shaded}{}{}
\newcommand{\AlertTok}[1]{\textcolor[rgb]{1.00,0.00,0.00}{\textbf{#1}}}
\newcommand{\AnnotationTok}[1]{\textcolor[rgb]{0.38,0.63,0.69}{\textbf{\textit{#1}}}}
\newcommand{\AttributeTok}[1]{\textcolor[rgb]{0.49,0.56,0.16}{#1}}
\newcommand{\BaseNTok}[1]{\textcolor[rgb]{0.25,0.63,0.44}{#1}}
\newcommand{\BuiltInTok}[1]{\textcolor[rgb]{0.00,0.50,0.00}{#1}}
\newcommand{\CharTok}[1]{\textcolor[rgb]{0.25,0.44,0.63}{#1}}
\newcommand{\CommentTok}[1]{\textcolor[rgb]{0.38,0.63,0.69}{\textit{#1}}}
\newcommand{\CommentVarTok}[1]{\textcolor[rgb]{0.38,0.63,0.69}{\textbf{\textit{#1}}}}
\newcommand{\ConstantTok}[1]{\textcolor[rgb]{0.53,0.00,0.00}{#1}}
\newcommand{\ControlFlowTok}[1]{\textcolor[rgb]{0.00,0.44,0.13}{\textbf{#1}}}
\newcommand{\DataTypeTok}[1]{\textcolor[rgb]{0.56,0.13,0.00}{#1}}
\newcommand{\DecValTok}[1]{\textcolor[rgb]{0.25,0.63,0.44}{#1}}
\newcommand{\DocumentationTok}[1]{\textcolor[rgb]{0.73,0.13,0.13}{\textit{#1}}}
\newcommand{\ErrorTok}[1]{\textcolor[rgb]{1.00,0.00,0.00}{\textbf{#1}}}
\newcommand{\ExtensionTok}[1]{#1}
\newcommand{\FloatTok}[1]{\textcolor[rgb]{0.25,0.63,0.44}{#1}}
\newcommand{\FunctionTok}[1]{\textcolor[rgb]{0.02,0.16,0.49}{#1}}
\newcommand{\ImportTok}[1]{\textcolor[rgb]{0.00,0.50,0.00}{\textbf{#1}}}
\newcommand{\InformationTok}[1]{\textcolor[rgb]{0.38,0.63,0.69}{\textbf{\textit{#1}}}}
\newcommand{\KeywordTok}[1]{\textcolor[rgb]{0.00,0.44,0.13}{\textbf{#1}}}
\newcommand{\NormalTok}[1]{#1}
\newcommand{\OperatorTok}[1]{\textcolor[rgb]{0.40,0.40,0.40}{#1}}
\newcommand{\OtherTok}[1]{\textcolor[rgb]{0.00,0.44,0.13}{#1}}
\newcommand{\PreprocessorTok}[1]{\textcolor[rgb]{0.74,0.48,0.00}{#1}}
\newcommand{\RegionMarkerTok}[1]{#1}
\newcommand{\SpecialCharTok}[1]{\textcolor[rgb]{0.25,0.44,0.63}{#1}}
\newcommand{\SpecialStringTok}[1]{\textcolor[rgb]{0.73,0.40,0.53}{#1}}
\newcommand{\StringTok}[1]{\textcolor[rgb]{0.25,0.44,0.63}{#1}}
\newcommand{\VariableTok}[1]{\textcolor[rgb]{0.10,0.09,0.49}{#1}}
\newcommand{\VerbatimStringTok}[1]{\textcolor[rgb]{0.25,0.44,0.63}{#1}}
\newcommand{\WarningTok}[1]{\textcolor[rgb]{0.38,0.63,0.69}{\textbf{\textit{#1}}}}


\usepackage{fvextra}
\DefineVerbatimEnvironment{Highlighting}{Verbatim}{
  commandchars=\\\{\},
  breaklines=true,
  breakanywhere=true,
  fontsize=\small,
  frame=single,
  framesep=6pt,
  xleftmargin=1.5cm,
  xrightmargin=0.5cm
}


\usepackage{longtable,booktabs,array}
\usepackage{calc} % for calculating minipage widths
% Correct order of tables after \paragraph or \subparagraph
\usepackage{etoolbox}
\makeatletter
\patchcmd\longtable{\par}{\if@noskipsec\mbox{}\fi\par}{}{}
\makeatother
% Allow footnotes in longtable head/foot
\IfFileExists{footnotehyper.sty}{\usepackage{footnotehyper}}{\usepackage{footnote}}
\makesavenoteenv{longtable}


\usepackage{graphicx}
\makeatletter
\def\maxwidth{\ifdim\Gin@nat@width>\linewidth\linewidth\else\Gin@nat@width\fi}
\def\maxheight{\ifdim\Gin@nat@height>\textheight\textheight\else\Gin@nat@height\fi}
\makeatother
% Scale images if necessary, so that they will not overflow the page
% margins by default, and it is still possible to overwrite the defaults
% using explicit options in \includegraphics[width, height, ...]{}
\setkeys{Gin}{width=\maxwidth,height=\maxheight,keepaspectratio}
% Set default figure placement to htbp
\makeatletter
\def\fps@figure{htbp}
\makeatother
\setlength{\emergencystretch}{3em} % prevent overfull lines
\providecommand{\tightlist}{%
  \setlength{\itemsep}{0pt}\setlength{\parskip}{0pt}}
\setcounter{secnumdepth}{-\maxdimen} % remove section numbering







\usepackage{styles/tesina}

\usepackage{bookmark}
\IfFileExists{xurl.sty}{\usepackage{xurl}}{} % add URL line breaks if available
\urlstyle{same}
\hypersetup{
  pdftitle={Il Suono e la Forma},
  pdfauthor={Mario Rossi},
  hidelinks,
  pdfcreator={LaTeX via pandoc}}


\begin{document}

  % ===================================================================
  % FRONTESPIZIO PERSONALIZZATO (con RIASSUNTO integrato)
  % ===================================================================
  \begin{titlepage}
      \centering
      \sffamily % Usa il font sans-serif (Arial) per tutta la pagina.
      % --- Intestazione: Arial 12pt ---
      {
          \normalsize
          Corsi Accademici di Musica Elettronica DCPL34 \\
          Conservatorio A. Casella, L'Aquila \par
      }
      \vspace{1cm} % Spazio tra intestazione e titolo
      % --- Titolo e Sottotitolo: Arial 14pt/12pt ---
      {
          \large\bfseries % 14pt Grassetto per il titolo
          \MakeUppercase{Il Suono e la Forma} \par
          % --- Logica per il Sottotitolo ---
                      {\normalsize\itshape Un'analisi della sintesi
granulare \par} % 12pt Corsivo per il sottotitolo
                }
      \vspace{.5cm} % Spazio tra titolo e autore
      % --- Dati Candidato: Arial 12pt ---
      {
          \normalsize
          di Mario Rossi \par
      }
      \vspace{.5cm}
      % --- Dati Esame: Arial 12pt ---
      {
          \normalsize
          esame di Sintesi e Campionamento \\
          15/07/2025 \par
      }
      \vspace{.5cm}
      % --- RIASSUNTO (Abstract) ---
      % Viene posizionato qui, prima della fine della pagina.
      % Usiamo un minipage per controllarne la larghezza e l'allineamento.
            \begin{minipage}{0.9\textwidth} % Crea un box largo il 90% della pagina
          \centering\large\bfseries RIASSUNTO \par % Titolo del riassunto
          \vspace{0.5cm}
          \normalsize\raggedright\mdseries % Testo del riassunto in 12pt, allineato a sinistra
          questo dovrebbe essere il sommario\ldots{}
      \end{minipage}
            \vspace*{1cm} % Spazio prima del fondo pagina
  \end{titlepage}


% questo è table of content, acronimo toc.
  



\section{GUIDA ALL'USO DI PANDOC}\label{guida-alluso-di-pandoc}

Questa sezione illustra come utilizzare le principali funzionalità di
Pandoc per la redazione della tesina: citazioni bibliografiche, tabelle,
equazioni matematiche e figure.

\begin{center}\rule{0.5\linewidth}{0.5pt}\end{center}

\section{1. CITAZIONI BIBLIOGRAFICHE}\label{citazioni-bibliografiche}

Le citazioni si basano sui file \texttt{.bib} presenti nella cartella
\texttt{docs/} (bibliografia.bib, discografia.bib, sitografia.bib).
Pandoc utilizza lo stile CSL specificato nel frontmatter per formattare
automaticamente le citazioni.

\subsection{1.1 Citazione semplice}\label{citazione-semplice}

Per citare un'opera in modo generico:

\begin{Shaded}
\begin{Highlighting}[]
\NormalTok{La sintesi digitale è stata ampiamente studiata \textasciigrave{}[@delduca1987]\textasciigrave{}.}
\end{Highlighting}
\end{Shaded}

\textbf{Risultato:} La sintesi digitale è stata ampiamente studiata (Del
Duca 1987).

\subsection{1.2 Citazioni multiple}\label{citazioni-multiple}

Per citare più opere contemporaneamente:

\begin{Shaded}
\begin{Highlighting}[]
\NormalTok{Diversi autori hanno affrontato il tema \textasciigrave{}([@delduca1987]; [@bianchini2000]; [@dannenberg2003])\textasciigrave{}.}
\end{Highlighting}
\end{Shaded}

\textbf{Risultato:} Diversi autori hanno affrontato il tema
({[}@delduca1987{]}; {[}@bianchini2000{]}; {[}@dannenberg2003{]}).

\subsection{1.3 Citazione con numero di
pagina}\label{citazione-con-numero-di-pagina}

Per citare una pagina specifica:

\begin{Shaded}
\begin{Highlighting}[]
\NormalTok{Come sottolinea Del Duca \textasciigrave{}[@delduca1987, p. 17]\textasciigrave{}, la sintesi granulare...}
\end{Highlighting}
\end{Shaded}

\textbf{Risultato:} Come sottolinea Del Duca {[}@delduca1987, p. 17{]},
la sintesi granulare\ldots{}

\subsection{1.4 Citazione testuale
(blocco)}\label{citazione-testuale-blocco}

Per riportare una citazione letterale estesa, usa il simbolo
\texttt{\textgreater{}}:

Come afferma Adorno {[}@adorno1959, p. 23{]}:

\begin{quote}
La musica moderna richiede un'attenzione particolare ai processi di
sintesi e trasformazione del suono. Questi processi definiscono
l'essenza stessa della composizione elettronica.
\end{quote}

\textbf{Nota:} Ogni riga della citazione deve iniziare con
\texttt{\textgreater{}}. La citazione apparirà rientrata nel PDF finale.

\begin{center}\rule{0.5\linewidth}{0.5pt}\end{center}

\section{2. TABELLE}\label{tabelle}

Pandoc supporta la creazione di tabelle con riferimenti incrociati. È
fondamentale includere sia la didascalia (caption) che l'identificatore
per poter richiamare la tabella nel testo.

\subsection{2.1 Sintassi base}\label{sintassi-base}

\begin{longtable}[]{@{}lll@{}}
\caption{Descrizione della
tabella}\label{tbl:identificatore}\tabularnewline
\toprule\noalign{}
a & b & c \\
\midrule\noalign{}
\endfirsthead
\toprule\noalign{}
a & b & c \\
\midrule\noalign{}
\endhead
\bottomrule\noalign{}
\endlastfoot
1 & 2 & 3 \\
4 & 5 & 6 \\
7 & 8 & 9 \\
\end{longtable}

\textbf{Componenti essenziali:}

\begin{itemize}
\item
  \textbf{Riga delle intestazioni:} \texttt{a\ \ b\ \ c} (separati da
  spazi o tab)
\item
  \textbf{Riga di separazione:} \texttt{-\/-\/-\ \ -\/-\/-\ \ -\/-\/-}
  (indica dove iniziano i dati)
\item
  \textbf{Righe di dati:} valori separati da spazi multipli
\item
  \textbf{Didascalia:} \texttt{:\ Testo\ della\ didascalia}
\item
  \textbf{Identificatore:} \texttt{\{\#tbl:nome\}} (per i riferimenti
  incrociati)
\end{itemize}

\subsection{2.2 Riferimento alla
tabella}\label{riferimento-alla-tabella}

Per richiamare la tabella nel testo:

\begin{Shaded}
\begin{Highlighting}[]
\NormalTok{Come si vede nella Tabella @tbl:identificatore, i risultati mostrano...}
\end{Highlighting}
\end{Shaded}

\textbf{Risultato:} Come si vede nella @tbl:identificatore, i risultati
mostrano\ldots{}

\subsection{2.3 Esempio completo}\label{esempio-completo}

\begin{longtable}[]{@{}lll@{}}
\caption{Parametri dell'oscillatore}\label{tbl:parametri}\tabularnewline
\toprule\noalign{}
Parametro & Valore & Unità \\
\midrule\noalign{}
\endfirsthead
\toprule\noalign{}
Parametro & Valore & Unità \\
\midrule\noalign{}
\endhead
\bottomrule\noalign{}
\endlastfoot
Frequenza & 440 & Hz \\
Durata & 2.5 & s \\
Ampiezza & 0.8 & dB \\
\end{longtable}

La Tabella @tbl:parametri mostra le impostazioni utilizzate
nell'esperimento.

\begin{center}\rule{0.5\linewidth}{0.5pt}\end{center}

\section{3. EQUAZIONI MATEMATICHE}\label{equazioni-matematiche}

Le equazioni utilizzano la sintassi LaTeX e possono essere referenziate
nel testo grazie a \texttt{pandoc-crossref}.

\subsection{3.1 Sintassi base}\label{sintassi-base-1}

\[
E = mc^2
\] \{\#eq:identificatore\}

\textbf{Componenti:}

\begin{itemize}
\item
  \textbf{Delimitatori:} \texttt{\$\$} (apertura) e \texttt{\$\$}
  (chiusura)
\item
  \textbf{Contenuto LaTeX:} formula matematica standard
\item
  \textbf{Identificatore:} \texttt{\{\#eq:nome\}} sulla stessa riga
  della chiusura
\end{itemize}

\subsection{3.2 Riferimento
all'equazione}\label{riferimento-allequazione}

L'Equazione @eq:identificatore mostra la relazione energia-massa\ldots{}

\textbf{Risultato:} L'Equazione (1) mostra la relazione
energia-massa\ldots{}

\subsection{3.3 Esempi comuni}\label{esempi-comuni}

\textbf{Formula semplice:}

\[
f(x) = x^2 + 2x + 1
\] \{\#eq:polinomio\}

\textbf{Formula con frazioni:}

\[
\frac{d}{dx}f(x) = \lim_{h \to 0} \frac{f(x+h) - f(x)}{h}
\] \{\#eq:derivata\}

\textbf{Sommatoria:}

\[
\sum_{i=1}^{n} x_i = x_1 + x_2 + \cdots + x_n
\] \{\#eq:somma\}

\textbf{Nota:} Per simboli e comandi avanzati, consulta la
documentazione LaTeX.

\begin{center}\rule{0.5\linewidth}{0.5pt}\end{center}

\section{4. FIGURE E IMMAGINI}\label{figure-e-immagini}

Esistono due metodi principali per inserire immagini: la sintassi Pandoc
(più semplice, con riferimenti automatici) e l'ambiente LaTeX
\texttt{wrapfigure} (per testo che scorre attorno all'immagine).

\subsection{4.1 Figure con Pandoc (metodo
consigliato)}\label{figure-con-pandoc-metodo-consigliato}

\textbf{Sintassi:}

\begin{Shaded}
\begin{Highlighting}[]
\NormalTok{![Didascalia dell\textquotesingle{}immagine](percorso/immagine.png)\{\#fig:identificatore width=60\%\}}
\end{Highlighting}
\end{Shaded}

\textbf{Componenti:}

\begin{itemize}
\item
  \texttt{!} indica che è un'immagine
\item
  \texttt{{[}Didascalia{]}} testo che apparirà sotto la figura
\item
  \texttt{(percorso)} path relativo al file (es.
  \texttt{docs/images/img001.png})
\item
  \texttt{\{\#fig:nome\}} identificatore per riferimenti incrociati
\item
  \texttt{width=XX\%} larghezza in percentuale della pagina
\end{itemize}

\textbf{Esempio:}

\begin{figure}
\centering
\includegraphics[width=0.7\linewidth,height=\textheight,keepaspectratio,alt={Spettrogramma del segnale analizzato}]{docs/images/img001.png}
\caption{Spettrogramma del segnale analizzato}\label{fig:spettro}
\end{figure}

La Figura @fig:spettro mostra la distribuzione delle frequenze nel
tempo.

\textbf{Vantaggi:} - Sintassi semplice e leggibile - Numerazione
automatica - Riferimenti incrociati funzionanti

\subsection{4.2 Wrapfigure (testo che scorre
attorno)}\label{wrapfigure-testo-che-scorre-attorno}

Per far scorrere il testo attorno a un'immagine, usa l'ambiente LaTeX
\texttt{wrapfigure}:

\begin{Shaded}
\begin{Highlighting}[]
\KeywordTok{\textbackslash{}begin}\NormalTok{\{}\ExtensionTok{wrapfigure}\NormalTok{\}\{r\}\{0.4}\FunctionTok{\textbackslash{}textwidth}\NormalTok{\}}
  \FunctionTok{\textbackslash{}centering}
  \BuiltInTok{\textbackslash{}includegraphics}\NormalTok{[width=0.38}\FunctionTok{\textbackslash{}textwidth}\NormalTok{]\{}\ExtensionTok{docs/images/img001.png}\NormalTok{\}}
  \FunctionTok{\textbackslash{}caption}\NormalTok{\{Diagramma esplicativo\}}
  \KeywordTok{\textbackslash{}label}\NormalTok{\{}\ExtensionTok{fig:diagramma}\NormalTok{\}}
\KeywordTok{\textbackslash{}end}\NormalTok{\{}\ExtensionTok{wrapfigure}\NormalTok{\}}
\end{Highlighting}
\end{Shaded}

\textbf{Parametri:} - \texttt{\{r\}} posizione: \texttt{r} (destra) o
\texttt{l} (sinistra) - \texttt{\{0.4\textbackslash{}textwidth\}}
larghezza dello spazio riservato all'immagine -
\texttt{{[}width=0.38\textbackslash{}textwidth{]}} larghezza effettiva
dell'immagine (leggermente inferiore) -
\texttt{\textbackslash{}label\{fig:nome\}} identificatore LaTeX (NON
usare \texttt{\#fig:})

\textbf{ATTENZIONE:} - Con wrapfigure devi inserire
\textbf{immediatamente dopo} un paragrafo di testo sufficientemente
lungo - Non funziona bene con elenchi puntati o altri elementi speciali
- Il riferimento si fa con \texttt{\textbackslash{}ref\{fig:nome\}} in
LaTeX, NON con \texttt{@fig:nome}

\textbf{Esempio d'uso corretto:}

\begin{Shaded}
\begin{Highlighting}[]
\KeywordTok{\textbackslash{}begin}\NormalTok{\{}\ExtensionTok{wrapfigure}\NormalTok{\}\{r\}\{0.35}\FunctionTok{\textbackslash{}textwidth}\NormalTok{\}}
  \FunctionTok{\textbackslash{}centering}
  \BuiltInTok{\textbackslash{}includegraphics}\NormalTok{[width=0.33}\FunctionTok{\textbackslash{}textwidth}\NormalTok{]\{}\ExtensionTok{docs/images/diagramma.png}\NormalTok{\}}
  \FunctionTok{\textbackslash{}caption}\NormalTok{\{Schema del processo\}}
  \KeywordTok{\textbackslash{}label}\NormalTok{\{}\ExtensionTok{fig:schema}\NormalTok{\}}
\KeywordTok{\textbackslash{}end}\NormalTok{\{}\ExtensionTok{wrapfigure}\NormalTok{\}}

\NormalTok{Questo paragrafo deve essere abbastanza lungo da permettere al testo di }
\NormalTok{scorrere correttamente attorno all\textquotesingle{}immagine. La sintesi granulare si basa }
\NormalTok{sulla scomposizione del segnale audio in brevi segmenti temporali chiamati }
\NormalTok{"grani", ognuno dei quali viene processato indipendentemente. Questa tecnica }
\NormalTok{permette di ottenere trasformazioni sonore complesse...}
\end{Highlighting}
\end{Shaded}

\subsection{4.3 Confronto: Pandoc vs
Wrapfigure}\label{confronto-pandoc-vs-wrapfigure}

\begin{longtable}[]{@{}lll@{}}
\caption{Descrizione della tabella}\label{tbl:tab}\tabularnewline
\toprule\noalign{}
Caratteristica & Pandoc \texttt{!{[}...{]}} & Wrapfigure \\
\midrule\noalign{}
\endfirsthead
\toprule\noalign{}
Caratteristica & Pandoc \texttt{!{[}...{]}} & Wrapfigure \\
\midrule\noalign{}
\endhead
\bottomrule\noalign{}
\endlastfoot
\textbf{Semplicità} & si Molto semplice & no Richiede LaTeX \\
\textbf{Riferimenti} & si \texttt{@fig:nome} & no
\texttt{\textbackslash{}ref\{fig:nome\}} \\
\textbf{Posizionamento} & Centro pagina & Testo attorno \\
\textbf{Quando usare} & Maggior parte dei casi & Solo se necessario \\
\end{longtable}

\textbf{Raccomandazione:} Usa la sintassi Pandoc (\texttt{!{[}...{]}})
come default. Ricorri a \texttt{wrapfigure} solo quando hai davvero
bisogno che il testo scorra attorno all'immagine.

\begin{center}\rule{0.5\linewidth}{0.5pt}\end{center}

\section{5. RIEPILOGO RAPIDO}\label{riepilogo-rapido}

\subsection{Citazioni}\label{citazioni}

\begin{itemize}
\tightlist
\item
  Semplice: \texttt{{[}@chiave{]}}
\item
  Con pagina: \texttt{{[}@chiave,\ p.\ 17{]}}
\item
  Multiple: \texttt{{[}@chiave1{]};\ {[}@chiave2{]}}
\item
  Blocco: righe che iniziano con \texttt{\textgreater{}}
\end{itemize}

\subsection{Tabelle}\label{tabelle-1}

\begin{Shaded}
\begin{Highlighting}[]
\NormalTok{col1    col2    col3}
\NormalTok{{-}{-}{-}     {-}{-}{-}     {-}{-}{-}}
\NormalTok{a       b       c}

\NormalTok{: Didascalia \{\#tbl:nome\}}

\NormalTok{Riferimento: \textasciigrave{}@tbl:nome\textasciigrave{}}
\end{Highlighting}
\end{Shaded}

\subsection{Equazioni}\label{equazioni}

\[
formula LaTeX
\] \{\#eq:nome\}

Riferimento: \texttt{@eq:nome}

\subsection{Figure}\label{figure}

\begin{figure}
\centering
\includegraphics[width=0.6\linewidth,height=\textheight,keepaspectratio,alt={Didascalia}]{docs/images/img001.png}
\caption{Didascalia}\label{fig:nome}
\end{figure}

Riferimento: @fig:nome

\begin{center}\rule{0.5\linewidth}{0.5pt}\end{center}

\section{6. CONSIGLI PRATICI}\label{consigli-pratici}

\begin{enumerate}
\def\labelenumi{\arabic{enumi}.}
\item
  \textbf{Identificatori univoci:} Usa nomi descrittivi e unici per ogni
  elemento (\texttt{\#fig:spettrogramma}, \texttt{\#tbl:risultati},
  \texttt{\#eq:fourier})
\item
  \textbf{Percorsi delle immagini:} Usa sempre percorsi relativi dalla
  root del progetto (es. \texttt{docs/images/foto.png})
\item
  \textbf{Dimensioni immagini:} Parti da \texttt{width=60\%} e aggiusta
  in base alle necessità (valori comuni: 40\%, 50\%, 70\%, 80\%)
\item
  \textbf{Riferimenti incrociati:} Usa sempre \texttt{@fig:},
  \texttt{@tbl:}, \texttt{@eq:} per riferimenti automatici che si
  aggiornano con la numerazione
\item
  \textbf{Bibliografia:} Assicurati che le chiavi nel file \texttt{.bib}
  corrispondano esattamente a quelle usate nelle citazioni
  (case-sensitive!)
\item
  \textbf{Testing:} Compila frequentemente il PDF per verificare che
  riferimenti e citazioni funzionino correttamente
\end{enumerate}

\begin{center}\rule{0.5\linewidth}{0.5pt}\end{center}

\section{Sezione di esempio con tutti gli
elementi}\label{sezione-di-esempio-con-tutti-gli-elementi}

Di seguito un esempio che integra tutti gli elementi:

La sintesi granulare, introdotta da Curtis Roads {[}@dannenberg2003{]},
si basa sulla manipolazione di brevi segmenti sonori. Come afferma Del
Duca {[}@delduca1987, p. 45{]}:

\begin{quote}
La granularità del suono permette un controllo microscopico della
materia sonora, aprendo possibilità compositive inedite.
\end{quote}

\begin{figure}
\centering
\includegraphics[width=0.5\linewidth,height=\textheight,keepaspectratio,alt={Rappresentazione schematica di un grano sonoro}]{docs/images/img001.png}
\caption{Rappresentazione schematica di un grano
sonoro}\label{fig:grano}
\end{figure}

La Figura @fig:grano illustra la struttura temporale di un singolo
grano. I parametri fondamentali sono riassunti nella Tabella
@tbl:param\_grano.

\begin{Shaded}
\begin{Highlighting}[]
\NormalTok{Parametro       Range           Descrizione}
\NormalTok{{-}{-}{-}{-}{-}{-}{-}{-}{-}{-}{-}     {-}{-}{-}{-}{-}{-}{-}{-}{-}{-}{-}     {-}{-}{-}{-}{-}{-}{-}{-}{-}{-}{-}{-}{-}{-}{-}{-}{-}{-}{-}{-}{-}{-}{-}{-}{-}{-}}
\NormalTok{Durata          5{-}100 ms        Lunghezza del grano}
\NormalTok{Inviluppo       Gaussiano       Forma della finestra}
\NormalTok{Densità         10{-}1000 Hz      Frequenza di emissione}
\end{Highlighting}
\end{Shaded}

: Parametri della sintesi granulare \{\#tbl:param\_grano\}

La relazione tra densità e sovrapposizione è espressa dall'Equazione
@eq:overlap:

\[
O = \frac{D \cdot T}{1000}
\] \{\#eq:overlap\}

dove \(O\) è il fattore di sovrapposizione, \(D\) la densità in Hz e
\(T\) la durata in millisecondi.

\section{CONCLUSIONE}\label{conclusione}

Eppur si muove\ldots{}

\section{BIBLIOGRAFIA}\label{bibliografia}

\protect\phantomsection\label{refs-bib}

\section{DISCOGRAFIA}\label{discografia}

\protect\phantomsection\label{refs-dis}

\section{SITOGRAFIA}\label{sitografia}

\protect\phantomsection\label{refs-sit}

\end{document}
